% MODELO CONEM 2016
\documentclass[10pt,fleqn,a4paper]{article}
\usepackage{abcm}
\begin{document}
    
    % CABEÇALHO
    \fancypagestyle{firststyle}
	{
   		\lhead{\emph{Anais do XXIII Encontro de Iniciação Científica e Pós-Graduação do ITA -  XXIII ENCITA / 2017 Instituto Tecnológico de Aeronáutica, São José dos Campos, SP, Brasil, XXIII  26 de outubro de 2017}}  
	}
    \thispagestyle{firststyle}
    \vspace{-.5cm}
    \hspace{-.8cm}
    \begin{tabular}{p{\textwidth}}
    \begin{center}
    \vspace{-.6cm}
    \title{Controle de motor elétrico Brushless Maxon 45fl-200142 para robô jogador de futebol Small Size}
    \end{center}
    \textbf{Aloysio Galvão Lopes}\\
    \small{Instituto Tecnológico de Aeronáutica}\\
    \small{Rua H8C, 322, DCTA}\\
    \small{12.228-462 - São José dos Campos/SP}\\
    \small{Bolsista PIBIC - CNPq}\\
    \small{aloysiogl@gmail.com}\\
    \\ 
    \textbf{Carlos César Aparecido Eguti}\\
    \small{Instituto Tecnológico de Aeronáutica}\\
    \small{Centro de Competência em Manufatura}\\
    \small{Praça Marechal Eduardo Gomes, 50}\\
    \small{12.229-900 – São José dos Campos / SP}\\
    \small{cesar.eguti@gmail.com}\\
    \\ 
    \textbf{Marcos Ricardo Omena de Albuquerque Maximo}\\
    \small{Instituto Tecnológico de Aeronáutica}\\
    \small{Laboratório de Sistemas Computacionais Autônomos (LAB-SCA)}\\
    \small{Divisão de Ciência da Computação}\\
    \small{Praça Marechal Eduardo Gomes, 50}\\
    \small{12.229-900 – São José dos Campos / SP}\\
    \small{maximo.marcos@gmail.com}\\
%    \authors{Nome do primeiro autor, e-mail$^1$} \\
%    \authors{Nome do segundo autor, e-mail$^1$} \\
%    \authors{Nome do terceiro autor, e-mail$^2$} \\\\
%    \institution{$^1$Nome da instituição, endereço para correspondência} \\
%    \institution{$^2$Nome da instituição, endereço para correspondência} \\
%    \\
%    \authors{\textcolor[rgb]{0.98,0.00,0.00}{Mesmo  formato para outros autores e instituições, se houver.}} \\
    \\
    \abstract{\textbf{Resumo:}
     Neste trabalho, é realizada a simulação e implementação do controle de um motor elétrico do tipo Brushless Maxon 45fl-200142. Os objetivos foram: realizar o modelamento matemático das equações que regem o motor citado; simular o modelo obtido; desenvolver, simular e otimizar os parâmetros de um controlador PI; desenvolver o hardware para o controle  e implementar o controlador para o motor.
     
     Para isso, foi, inicialmente, desenvolvido o hardware para controle do motor. Em seguida, o hardware foi testado em um motor BLDC de drive de DVD em loop aberto. Após isso, foi realizada a modelagem do motor baseada nos parâmetros físicos fornecidos pelo fabricante; com base nisso, foi possível simular o comportamento do motor e projetar o controlador. Por fim, foi possível implementar o controlador no hardware desenvolvido e controlar o motor Maxon 45fl-200142.
     
     Dos resultados obtidos, foi possível concluir que o controle do motor só pode ser realizado com feedback. Logo, o controle do motor do drive de DVD em loop aberto não obteve sucesso. No entanto, foi possível realizar o controle do motor Maxon devido ao feedback de sensores de efeito Hall, além do mais os resultados de velocidade angular obtidos pelo motor foram fiéis ao modelo.}\\    
     \keywords{\textbf{Palavras-chave:} Motor elétrico Brushless, Modelagem física, Simulação, Controle de velocidade.}\\
    \end{tabular}
    

    \section{Introdução}
    A ITAndroids, equipe de robótica do ITA, é uma iniciativa de alunos do ITA que representa a instituição em diversas competições de robótica nacionais e internacionais. O principal objetivo dessa iniciativa é a integração dos alunos com atividades de pesquisa em engenharia, em especial em robótica e inteligência artificial.
    
    A ITAndroids acredita que o hardware é fundamental no desenvolvimento de qualquer projeto de robótica, por isso, recentemente tem-se investido bastante no desenvolvimento dessa área. O desenvolvimento de conhecimento na área de controle de motores elétricos, assim, torna-se fundamental para a melhoria técnica da iniciativa.
    
    Nesse sentido, abre-se caminho para a participação na categoria Small Size descrita em \cite{robocupregras}, a qual necessita de uma eletrônica mais desenvolvida o que implica diretamente no desenvolvimento de conhecimento sobre controle de motores. Os motores utilizados para a locomoção dos robôs são motores Brushless, essa escolha se justifica pelo fato de que motores Brushless (sem comutadores mecânicos do tipo escova) são mais eficientes, ocupam menos espaço e apresentam desgaste muito menor que motores DC tradicionais.
    
    Uma vez que o desgaste mecânico durante as partidas de futebol de robôs Small Size é elevado e é preciso alto rendimento dos sistemas mecânicos: os motores Brushless, também chamados BLDC são a escolha ideal; além disso, as principais equipes internacionais fazem uso de desses motores. Como principal exemplo temos a equipe tailandesa SKUBA, descrita em \cite{skuba}, na qual se baseia o projeto mecânico do Small Size da ITAndroids. Em contrapartida, o controle deste tipo de atuador mecânico se dá de maneira mais complexa, uma vez que a comutação do campo magnético não é feita mecanicamente por escovas. A Figura \ref{fig:esquemamotor}, abaixo, ilustra a mecânica de um motor BLDC.

	\begin{figure}[ht]
		\begin{center}
			\includegraphics[angle=0, scale=0.4]{images/4-Pole-brushless-DC-motor-animation}
		\end{center}
		\caption{\textbf{Esquema completo de um motor elétrico Brushless trifásico. Retirado de \cite{figuramotor}.}}
		\label{fig:esquemamotor}
	\end{figure}
    
    O controle da comutação do campo magnético para esse tipo de motor é feito eletronicamente e em malha fechada com explicado em \cite{ogata1982engenharia}, seguindo o procedimento descrito em \cite{introducaobldc}, o que acarreta que seja necessário que um dispositivo eletrônico, tal como um microcontrolador, fique dedicado ao controle de cada um dos quatro motores do robô Small Size. Isso aumenta a complexidade do sistema de locomoção como um todo, no entanto traz diversos ganhos em eficiência para o projeto.
    
    Nesse sentido, busca-se, aqui, modelar o motor Maxon 45fl-200142 e ser capaz de realizar a comutação eletronicamente. Adicionalmente, deseja-se, simular e implementar um controlador PI.
    
    \section{Modelagem do motor}
    \subsection{Funcionamento do motor elétrico Brushless Maxon 45fl-200142}
    
    Um motor elétrico Brushless têm várias possíveis configurações, no entanto, a configuração trifásica ganha destaque em eficiência. Isso pode ser constatado pela grande preferência no mercado por tal tipo de motor BLDC. O motor aqui estudado é de configuração trifásica, logo, há três entradas cada uma para uma bobina do motor.
    
    Isso significa que existem três fases diferentes que devem ser eletronicamente comutadas para que se obtenha a rotação do motor. Além disso, o ciclo de rotação do motor é composto de seis etapas distintas, portanto o motor rotaciona com uma comutação de seis passos. Há apenas uma sequência correta de seis combinações das fases do motor para cada sentido de rotação. Pode-se, no entanto, encontrar sequências de comutação distintas das duas principais que rotacionam o motor. Comutações com essas sequências resultam em menor eficiência e perda de precisão na rotação, por isso, é essencial encontrar as sequências corretas de comutação.
    
    Vale destacar, como já dito, que cada fase corresponde a uma bobina do motor e que estas bobinas estão interligadas. Assim, cada passo corresponde a aplicar tensão entre dois terminais de duas bobinas. As bobinas podem estar distribuídas ao redor da carcaça do motor como pode ser notado na Fig. \ref{fig:esquemamotor}, o que faz que uma rotação mecânica completa não seja equivalente a um ciclo elétrico (ciclização entre os seis passos de comutação). No caso do motor estudado, são necessárias 48 comutações para que uma revolução completa seja atingida; isso, portanto, significa que a cada oito ciclos elétricos ocorre um ciclo mecânico no motor. A Figura \ref{fig:comutacao}, abaixo, ilustra a comutação de fases em um motor BLDC.

	\begin{figure}[ht]
		\begin{center}
			\includegraphics[angle=0, scale=0.5]{images/comutacao}
		\end{center}
		\caption{\textbf{Contra eletromotriz nas três fases de um motor BLDC comutando. Retirado de \cite{introducaobldc}.}}
		\label{fig:comutacao}
	\end{figure}
    
    
    Adicionalmente, é comum haver três sensores de efeito Hall (como é o caso deste motor), para que seja possível controlar a comutação com feedback. A combinação das três saídas de cada um dos sensores de efeito Hall dá a próxima comutação do motor de maneira única para cada sentido de rotação. Com isso, pode-se montar uma tabela de correspondências entre saídas do sensores de efeito Hall e passo de comutação: a tabela de comutação do motor. Essa tabela, em geral, não é fornecida pelo fabricante e deve ser montada para que seja possível controlar o motor.
    
    \subsection{Modelamento do motor}
    
    O modelo matemático utilizado no estudo do motor BLDC é o mesmo modelo utilizado para um motor DC convencional, o qual, em termos práticos, é uma boa aproximação. Um modelamento completo pode ser encontrado em \cite{modelomotor}. \cite{livrocarta} descreve mais detalhadamente o controle de motor DC. Considera-se que o torque produzido pelo motor é proporcional à corrente como mostrado em Eq. (\ref{torquemotor}) e que a tensão no motor é proporcional à velocidade angular $\dot{\theta}$, como mostrado em Eq. (\ref{tensaomotor}).
    
    \begin{equation}
    \tau = K_ti \label{torquemotor}
    \end{equation}
    \begin{equation}
    V = K_b\dot{\theta} \label{tensaomotor}
    \end{equation}
    
    Considerando, ainda, o momento de inércia do motor $J$ e que existe um atrito viscoso $b\dot{\theta}$, pela terceira lei de Newton, chega-se à Eq. (\ref{leidenewtonmotor}). Utilizando a lei das malhas e considerando a queda de tensão devido à resistência do motor, à sua indutância e à conversão da energia em energia mecânica, sendo $\varepsilon$ a tensão da fonte, chega-se à Eq. (\ref{malhasmotor}).
    
    \begin{equation}
    J\ddot{\theta} = K_ti - b\dot{\theta} \label{leidenewtonmotor}
    \end{equation}
    \begin{equation}
    \varepsilon = K_b\dot{\theta} + L\frac{di}{dt} + Ri \label{malhasmotor}
    \end{equation}
    
    Pode-se, associando Eq. (\ref{leidenewtonmotor}) e Eq. (\ref{malhasmotor}), chegar à função de transferência do motor, mostrada abaixo em Eq. (\ref{plantamotor}). Observa-se que foram utilizados os dados fornecidos em \cite{Datasheet} pelo fabricante e foram desconsiderados os termos do atrito viscoso, pois eram negligenciáveis.
    
    \begin{equation}
    \frac{\dot{\Theta}(s)}{\varepsilon(s)} = \frac{13.095}{2.66 \cdot 10^{-6} s^2+0.0171s+1} \label{plantamotor}
    \end{equation}
    
    \section{Hardware}
    
    Foi confeccionado um hardware para acionamento do motor baseado no esquema mostrado abaixo, em Fig \ref{fig:diagrama}, descrito em \cite{atmeldiagrama}. 
    
	\begin{figure}[ht]
		\begin{center}
			\includegraphics[angle=0, scale=0.5]{images/circuitdiagram}
		\end{center}
		\caption{\textbf{Diagrama do circuito para acionamento do motor. Retirado de \cite{atmeldiagrama}.}}
		\label{fig:diagrama}
	\end{figure}
    
    Foram utilizados seis pares darlington TIP 122 para montar o esquema mostrado acima de três meias pontes H e uma placa de desenvolvimento Arduino$^{\small{TM}}$ Mega ADK para o controle. O sistema inicialmente foi testado com um motor Brushless retirado de um drive de DVD em loop aberto. Observou-se que sem um feedback dos sensores a velocidade máxima atingida pelo motor era muito pequena e a rotação não se dava de maneira suave. Isso mostra que sem informações sobre a posição do rotor, este facilmente não é capaz de acompanhar a variação do campo magnético.
    
    Abaixo, é mostrada na Figura \ref{fig:montagemcircuito} a montagem do circuito para o controle do motor. Na Figura \ref{fig:motorantigo}, é mostrado o motor de drive de DVD utilizado inicialmente.

	\begin{figure}[ht]
		\begin{center}
			\includegraphics[angle=0, scale=0.06]{images/montagemcircuito}
		\end{center}
		\caption{\textbf{Figura da montagem do circuito para controle do motor.}}
		\label{fig:montagemcircuito}
	\end{figure}

	\begin{figure}[ht]
		\begin{center}
			\includegraphics[angle=0, scale=0.06]{images/motorantigo}
		\end{center}
		\caption{\textbf{Motor de drive de DVD utilizado nos testes, nota-se que não há saídas para sensores de efeito Hall.}}
		\label{fig:motorantigo}
	\end{figure}
    
    Em uma segunda etapa, utilizou-se o motor Maxon 45fl-200142, cujo controle é o objetivo deste estudo. Nesse caso, foi possível utilizar o feedback dos sensores de efeito Hall e obter uma rotação suave, além disso, foi desenvolvido um script para a obtenção da tabela de comutação, uma vez que esta não é fornecida pelo fabricante. O que este script faz é ciclizar lentamente (1s) as saídas do motor conforme descrito na sequência de comutação direta em \cite{atmeldiagrama} e, em seguida, fazer a leitura dos sensores de efeito Hall. Com isso, é possível montar a tabela de comutação direta; a tabela inversa é montada por meio da tabela direta, apenas invertendo a lógica de rotação.
    
    Este script pode ser, posteriormente, utilizado para a obtenção da tabela de comutação de qualquer motor Brushless com sensores de efeito Hall. Vale ressaltar que a regulação da velocidade aqui é feita utilizando PWM. Para isso, uma saída PWM é gerada para cada transistor que leva corrente do motor ao ground, seguindo o esquema mostrado em Fig. \ref{fig:diagrama}.
    A Figura \ref{fig:motornovo}, abaixo, mostra o motor da Maxon rotacionando. Mais adiante serão tratados os aspectos de controle implementados.

	\begin{figure}[ht]
		\begin{center}
			\includegraphics[angle=0, scale=0.06]{images/motornovo}
		\end{center}
		\caption{\textbf{Motor Maxon 45fl-200142 rotacionando.}}
		\label{fig:motornovo}
	\end{figure}

	\newpage
    
    \section{Simulação}
    
    Foi implementada em MATLAB$^{\small{TM}}$ a função de transferência de Eq. (\ref{plantamotor}) e, com isso, foi possível obter o gráfico para a resposta em degrau do sistema, bem como o diagrama de Bode. As figuras Fig. \ref{fig:respostaemdegrau} e Fig \ref{fig:bode}, respectivamente, representam a resposta em degrau unitário e o diagrama de Bode da planta obtidos.

	\begin{figure}[ht]
		\begin{center}
			\includegraphics[angle=0, scale=0.7]{images/stepResponse}
		\end{center}
		\caption{\textbf{Análise da resposta à um degrau unitário da planta de Eq. (\ref{plantamotor}).}}
		\label{fig:respostaemdegrau}
	\end{figure}

	\begin{figure}[ht]
		\begin{center}
			\includegraphics[angle=0, scale=0.7]{images/bode}
		\end{center}
		\caption{\textbf{Diagrama de bode para a planta de Eq. (\ref{plantamotor}).}}
		\label{fig:bode}
	\end{figure}
    
    Foi utilizado um controlador PI (proporcional e integral); esta escolha se deve ao fato de que o termo integral do controlador elimina os erros em regime. Ademais, o controle de velocidade não exige tempo de resposta elevado (como seria o caso de um controlador PD). A função de transferência em tempo contínuo para o controlador PI pode ser vista na Eq. (\ref{tfcontroladorct}).
    
    \begin{equation}
    \frac{\varepsilon(s)}{\epsilon(s)}= Kp + \frac{Ki}{s} \label{tfcontroladorct}
    \end{equation}
    
    Foi realizada uma otimização com auxílio da ferramenta de PI tunning, presente no MATLAB$^{\small{TM}}$. Buscou-se obter um pequeno overshoot e cerca de 0.001 segundo até atingir o estado estacionário. No domínio da frequência, é possível ver as fases de margem e de ganho em Fig. \ref{fig:margin}. As constantes obtidas foram: Kp = 2.8522 e Ki = 104.55. Após isso, foi gerado o gráfico de Fig. \ref{fig:saidasimulada} para a saída em velocidade angular do sistema mostrado em Fig.  \ref{fig:sistemasimulink}.
    
    \begin{figure}[ht]
    	\begin{center}
    		\includegraphics[angle=0, scale=0.4]{images/sistemaEControlador}
    	\end{center}
    	\caption{\textbf{Sistema completo modelado com auxílio do SIMULINK$^{\small{\textbf{TM}}}$.}}
    	\label{fig:sistemasimulink}
    \end{figure}

	\begin{figure}[ht]
		\begin{center}
			\includegraphics[angle=0, scale=0.8]{images/margins.png}
		\end{center}
		\caption{\textbf{Margens de estabilidade do sistema em estudo.}}
		\label{fig:margin}
	\end{figure}

	\begin{figure}[ht]
		\begin{center}
			\includegraphics[angle=0, scale=0.6]{images/sistemStep}
		\end{center}
		\caption{\textbf{Saída de velocidade angular com valor fixado em 100rad/s.}}
		\label{fig:saidasimulada}
	\end{figure}
    
    Após a discretização, utilizando o método de Tustin, foi possível chegar à função de transferência discreta, mostrada em Eq. (\ref{tfdiscreta}), a qual possibilitou a implementação do controle seguindo a relação de recorrência mostrada em Eq. (\ref{recorrencia}).
    
    \begin{equation}
    \frac{\varepsilon(z)}{\epsilon(z)}= Kp + \frac{Ki T}{2}\frac{z+1}{z-1} \label{tfdiscreta}
    \end{equation}
    
    \begin{equation}
    \varepsilon[z] = \varepsilon[z-1] + Kp\cdot\epsilon[z] - Kp\cdot\epsilon[z-1] - \frac{Ki T}{2}(\epsilon[z] + \epsilon[z-1]) \label{recorrencia}
    \end{equation}
    
    \newpage
    
    \section{Considerações finais}
    
    Após a implementação do algoritmo, observou-se que o motor era capaz de atingir a velocidade de 99.2 radianos por segundo, o que mostra que o modelo se adéqua aos parâmetros físicos do sistema.
    
    Ressalta-se, por fim, que existem alguns melhoramentos a serem realizados no controlador antes da implementação final no robô. Pode-se incluir uma malha de corrente, uma vez que o hardware do robô permite. Além disso, deve-se contar a inércia do robô no modelo final a ser utilizado e seria interessante analisar a influência de outros efeitos na simulação como a discretização do controlador, atrasos presentes na malha, atrito seco e a quantização do encoder. Os testes realizados no sistema, também, devem ser refeitos no hardware final.
    
    Após a implementação do controlador aqui desenvolvida os próximos passos são de implementação no robô. Pretende-se implementar o controlador em uma FPGA, a qual está presente na placa principal do robô. O robô terá sua estrutura mecânica finalizada e será feia a integração dos sistemas eletrônicos e mecânicos, bem como a comunicação do hardware produzido com o computador.
 
    
    Os motores já foram posicionados na carcaça do primeiro protótipo do robô Small Size como mostrado em Fig. (\ref{fig:smallfoto}).

	\begin{figure}[ht]
		\begin{center}
			\includegraphics[angle=0, scale=0.3]{images/smallfoto}
		\end{center}
		\caption{\textbf{Fotografia do primeiro protótipo do robô Small Size com os quatro motores posicionados e operacionais.}}
		\label{fig:smallfoto}
	\end{figure}

	\section{Agradecimentos}
	
	Agradeço ao CNPq, pelo apoio financeiro durante a realização do projeto, bem como ao meu professor orientador pelo apoio à realização do projeto. Agradeço à iniciativa ITAndroids, por fornecer os treinamentos necessários e recursos materiais para a realização do projeto.
	
	\section{Referências}
	\nocite{Matlab}
	
	\bibliographystyle{abcm}
	\bibliography{bibliografia}
    
\end{document}