\documentclass[a4paper,11pt]{article}

\usepackage[english,brazil]{babel}
\usepackage[utf8]{inputenc}
\usepackage{amsmath}
\usepackage{algorithm}
\usepackage{algpseudocode}
\usepackage{graphicx}
\usepackage{fancyhdr}
\usepackage{url}
\renewcommand{\baselinestretch}{1.5}
\usepackage{titling}
\usepackage{geometry}
\geometry{
	a4paper,
	left=25mm,
	right=25mm,
	top=1in,
	bottom=1in,
}
\usepackage{amsmath}
\usepackage{amssymb}
\usepackage{subfig}
\usepackage{multirow}
\usepackage[table]{xcolor}
\usepackage[backend=bibtex,style=ieee,sorting=none]{biblatex}
\bibliography{bibliography.bib}

\newcolumntype{C}{>{\centering\arraybackslash}p{1em}}

\title{\textbf{Controle de motor elétrico Brushless para robô jogador de futebol Small Size}}

\author{Aloysio Galvão Lopes, Carlos Eguti\\
	Instituto Tecnológico de Aeronáutica\\
	São José dos Campos, SP, Brasil} %\\

\date{\today}

% ADD THE FOLLOWING COUPLE LINES INTO YOUR PREAMBLE
\let\OLDthebibliography\thebibliography
\renewcommand\thebibliography[1]{
	\OLDthebibliography{#1}
	\setlength{\parskip}{0pt}
	\setlength{\itemsep}{0pt plus 0.3ex}
}


% O DOCUMENTO COME�A AQUI
\begin{document}
	
	\tableofcontents
	\newpage

	\maketitle
	
	\section{Plano Inicial}
	\textbf{1º bimestre (ago/set)} %Conferir
	\begin{itemize}
		\item Estudar controle de formal geral e estudar os mecanismos de controle de motores Brushless.
	\end{itemize}
	\textbf{2º bimestre (out/nov)}
	\begin{itemize}
		\item Estabelecer o sistema físico e projetar o mecanismo eletrônico para controle do motor.
		\item  Realizar simulação de tal mecanismo.
	\end{itemize}
	\textbf{3º bimestre (dez/jan)}
	\begin{itemize}
		\item Estabelecer método de testes eficiente e confeccionar o mecanismo de controle do motor.
		\item Confecção do relatório parcial.
	\end{itemize}
	\textbf{4º bimestre (fev/mar)} %Conferir
	\begin{itemize}
		\item Realizar testes iniciais e buscar otimizações no sistema de controle confeccionado.
	\end{itemize}
	\textbf{5º bimestre (abr/mai)}
	\begin{itemize}
		\item Projetar um novo sistema eletrônico compatível com as características do protótipo do robô Small Size da ITAndroids.
	\end{itemize}
	\textbf{6º bimestre (jun/jul)}
	\begin{itemize}
		\item Implementar o sistema de controle no protótipo do Small Size da ITAndroids.
		\item Corrigir erros e buscar otimizações.
		\item Confecção do relatório parcial.
	\end{itemize}

	\section{Atividades realizadas}
	\textbf{(ago/set/out/nov/dez/jan)}
	\begin{itemize}
		\item Bolsa ainda não implementada
	\end{itemize}
	\textbf{(fev/mar)}
	\begin{itemize}
		\item Início das atividades e replanejamento devido à redução do tempo de vigência da bolsa.
		\item Realização do curso de controle aplicado fornecido pelo grupo de robótica ITAndroids.
		\item Estudo dos métodos de acionamento dos motores Brushless e do funcionamento mecênico dos motores BLDC o dos motores de passo.
	\end{itemize}
		\textbf{(abr/mai)}
	\begin{itemize}
		\item Continuidade do curso de controle e estudo do acionamento dos motores BLDC por meio de encoders e sensores de efeito Hall e estudo de técnicas para a montagem da tabela de comutação de um motor BLDC.
		\item Realização do projeto e montagem do hardware para o primeiro teste de rotação do um motor (com BLDC retirado de drive de CD) em loop aberto e realização do teste de rotação.
		\item Estudo da modelagem física de motores elétricos DC e compreensão das diferenças da modelagem de um motor BLDC.
	\end{itemize}
		\textbf{(jun/jul)}
		\begin{itemize}
			\item Adaptação do hardware para o teste de controle do motor Maxon 45fl-200142, utilizado no robô Small Size e realização de seu controle com o auxílio dos sensores de efeito hall, bem como integração inicial à estrutura mecânica do robô.
			\item Modelagem do motor, desenvolvimento de um controlador PID e posterior simulação por meio do software MATLAB \cite{MATLAB}.
			\item Implementação do controlador PID em C++ para teste no motor e coleta dos resultados finais do projeto.
			\item Confecção do relatório final.
		\end{itemize}
	\section{Descrição do Problema}
	
	
	
	\section{Resultados}
	
	\section{Conclusão}
	
	\section{Agradecimentos}
	
	\section{Bibliografia}
	%\bibliographystyle{IEEEtran}
	%\bibliographystyle{bibstyle}
	\printbibliography[heading=none]
	
\end{document}