\documentclass[a4paper,11pt]{article}

\usepackage[english,brazil]{babel}
\usepackage[utf8]{inputenc}
\usepackage{amsmath}
\usepackage{algorithm}
\usepackage{algpseudocode}
\usepackage{graphicx}
\usepackage{fancyhdr}
\usepackage{url}
\renewcommand{\baselinestretch}{1.5}
\usepackage{titling}
\usepackage{geometry}
\usepackage{indentfirst}
\geometry{
	a4paper,
	left=25mm,
	right=25mm,
	top=1in,
	bottom=1in,
}
\usepackage{amsmath}
\usepackage{amssymb}
\usepackage{subfig}
\usepackage{multirow}
\usepackage[table]{xcolor}
\usepackage[backend=bibtex,style=ieee,sorting=none]{biblatex}
\bibliography{bibliography.bib}

\newcolumntype{C}{>{\centering\arraybackslash}p{1em}}

\title{\textbf{Controle de motor elétrico Brushless para robô jogador de futebol Small Size}}

\author{Aloysio Galvão Lopes, Carlos Eguti\\
	Instituto Tecnológico de Aeronáutica\\
	São José dos Campos, SP, Brasil} %\\

\date{\today}

% ADD THE FOLLOWING COUPLE LINES INTO YOUR PREAMBLE
\let\OLDthebibliography\thebibliography
\renewcommand\thebibliography[1]{
	\OLDthebibliography{#1}
	\setlength{\parskip}{0pt}
	\setlength{\itemsep}{0pt plus 0.3ex}
}


% O DOCUMENTO COME�A AQUI
\begin{document}
	
	\tableofcontents
	\newpage

	\maketitle
	
	\section{Plano Inicial}
	\textbf{1º bimestre (ago/set)} %Conferir
	\begin{itemize}
		\item Estudar controle de formal geral e estudar os mecanismos de controle de motores Brushless.
	\end{itemize}
	\textbf{2º bimestre (out/nov)}
	\begin{itemize}
		\item Estabelecer o sistema físico e projetar o mecanismo eletrônico para controle do motor.
		\item  Realizar simulação de tal mecanismo.
	\end{itemize}
	\textbf{3º bimestre (dez/jan)}
	\begin{itemize}
		\item Estabelecer método de testes eficiente e confeccionar o mecanismo de controle do motor.
		\item Confecção do relatório parcial.
	\end{itemize}
	\textbf{4º bimestre (fev/mar)} %Conferir
	\begin{itemize}
		\item Realizar testes iniciais e buscar otimizações no sistema de controle confeccionado.
	\end{itemize}
	\textbf{5º bimestre (abr/mai)}
	\begin{itemize}
		\item Projetar um novo sistema eletrônico compatível com as características do protótipo do robô Small Size da ITAndroids.
	\end{itemize}
	\textbf{6º bimestre (jun/jul)}
	\begin{itemize}
		\item Implementar o sistema de controle no protótipo do Small Size da ITAndroids.
		\item Corrigir erros e buscar otimizações.
		\item Confecção do relatório parcial.
	\end{itemize}

	\section{Atividades realizadas}
	\textbf{(ago/set/out/nov/dez/jan)}
	\begin{itemize}
		\item Bolsa ainda não implementada
	\end{itemize}
	\textbf{(fev/mar)}
	\begin{itemize}
		\item Início das atividades e replanejamento devido à redução do tempo de vigência da bolsa.
		\item Realização do curso de controle aplicado fornecido pelo grupo de robótica ITAndroids.
		\item Estudo dos métodos de acionamento dos motores Brushless e do funcionamento mecênico dos motores BLDC o dos motores de passo.
	\end{itemize}
		\textbf{(abr/mai)}
	\begin{itemize}
		\item Continuidade do curso de controle e estudo do acionamento dos motores BLDC por meio de encoders e sensores de efeito Hall e estudo de técnicas para a montagem da tabela de comutação de um motor BLDC.
		\item Realização do projeto e montagem do hardware para o primeiro teste de rotação do um motor (com BLDC retirado de drive de CD) em loop aberto e realização do teste de rotação.
		\item Estudo da modelagem física de motores elétricos DC e compreensão das diferenças da modelagem de um motor BLDC.
	\end{itemize}
		\textbf{(jun/jul)}
		\begin{itemize}
			\item Adaptação do hardware para o teste de controle do motor Maxon 45fl-200142, utilizado no robô Small Size e realização de seu controle com o auxílio dos sensores de efeito hall, bem como integração inicial à estrutura mecânica do robô.
			\item Modelagem do motor, desenvolvimento de um controlador PID e posterior simulação por meio do software MATLAB \cite{MATLAB}.
			\item Implementação do controlador PID em C++ para teste no motor e coleta dos resultados finais do projeto.
			\item Confecção do relatório final.
		\end{itemize}
	\section{Descrição do Problema}
		A ITAndroids, equipe de robótica do ITA, é uma iniciativa de alunos do ITA que representa a instituição em diversas competições de robótica nacionais e internacionais. O principal objetivo dessa iniciativa é a integração dos alunos com atividades de pesquisa em engenharia, em especial em robótica e inteligência artificial.
		
		A ITAndroids acredita que o hardware é fundamental no desenvolvimento de qualquer projeto de robótica, por isso, recentemente tem-se investido bastante no desenvolvimento dessa área. O desenvolvimento de conhecimento na área de controle de motores elétricos, assim, torna-se fundamental para a melhoria técnica da iniciativa.
		
		Nesse sentido, abre-se caminho para a participação na categoria Small Size, a qual necessita de uma eletrônica mais desenvolvida o que implica diretamente o desenvolvimento de conhecimento sobre controle de motores. Os motores utilizados para a locomoção dos robôs são motores Brushless, essa escolha se justifica pelo fato de que motores Brushless são mais eficientes, ocupam menos espaço e apresentam desgaste muito menor que motores DC tradicionais.
		
		Uma vez que o desgaste mecânico durante as partidas de futebol de robôs Small Size é elevado e é preciso alto rendimento dos sistemas mecânicos os motores BLDC são a escolha ideal; além disso, as principais equipes internacionais, tal como a equipe tailandesa SKUBA \cite{skuba} fazem uso de desses motores. Em contrapartida, o controle deste tipo de atuador mecânico se dá de maneira mais complexa uma vez que a comutação do campo magnético não é feita mecanicamente por escovas. A figura \ref{fig:esquemamotor}, abaixo, ilustra a mecânica de um motor BLDC.
		
		\begin{figure}[ht]
			\centering
			\includegraphics[width=0.7\linewidth]{images/4-Pole-brushless-DC-motor-animation}
			\caption{Esquema de um motor elétrico Brushless.}
			\label{fig:esquemamotor}
		\end{figure}
		
		
		O controle da comutação do campo magnético para esse tipo de motor é feito eletronicamente, o que acarreta a necessidade de um dispositivo eletrônico tal como um microcontrolador \cite{introducaobldc}. Isso aumenta a complexidade do sistema de locomoção como um todo, no entanto traz diversos ganhos em eficiência para o projeto.
		
		Nesse sentido o objetivo principal deste trabalho é compreender o funcionamento de um motor elétrico Brushless e ser capaz de realizar a comutação eletronicamente. Adicionalmente, deseja-se modelar o motor, simular e implementar um controlador PID para um motor Maxon 45fl-200142, o qual é utilizado para a locomoção de um robô Small Size.
	
	\section{Resultados}
		\subsection{Funcionamento de um motor elétrico Brushless}
		
		\subsection{Modelo utilizado no estudo}
		
		\subsection{Hardware confeccionado para acionamento do motor}
		
		\subsection{Simulação do modelo}
		
		\subsection{Análise do resultados no motor Maxon 45fl-200142}
	\section{Conclusão}
	
	\section{Agradecimentos}
	Agradeço ao CNPQ, pelo apoio financeiro durante a realização do projeto, bem como ao meu professor orientador pelo apoio à realização do projeto. Agradeço à iniciativa ITAndroids, por fornecer os treinamentos necessários e recursos materiais para a realização do projeto.
	
	\section{Bibliografia}
	%\bibliographystyle{IEEEtran}
	%\bibliographystyle{bibstyle}
	\printbibliography[heading=none]
	
\end{document}